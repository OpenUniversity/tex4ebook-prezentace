\documentclass{article} 
\usepackage[english]{babel} 
\usepackage[T1]{fontenc}
\usepackage[utf8]{inputenc}
\title{Justification example}
\author{Michal Hoftich}
\usepackage{hyperref}

\begin{document} 

\maketitle

This is a example of \textit{HTML file} justification using \verb|tex4ht| and
\verb|Calzone|\footnote{\url{https://github.com/simoncozens/calzone}}. Text is
  copied from Wikipedia article about Prague

Prague () is the
capital and largest city of the Czech Republic. It is the fifteenth-largest
city in the European Union. It is also the historical capital of Bohemia.
Situated in the north-west of the country on the Vltava River, the city is home
to about 1.24 million people, while its larger urban zone is estimated to have
a population of nearly 2 million. The city has a temperate climate, with
warm summers and chilly winters. The origin of the name Praha is rather
associated with the word prah (that means a 'threshold'), which is a rapid on
the river. There is also the lowest unemployment rate in the
entire European Union.

Prague has been a political, cultural, and economic centre of central Europe
with waxing and waning fortunes during its 1,100-year existence. Founded during
the Romanesque and flourishing by the Gothic and Renaissance eras, Prague was
not only the capital of the Czech state, but also the seat of two Holy Roman
Emperors and thus also the capital of the Holy Roman Empire. It was an
important city to the Habsburg Monarchy and its Austro-Hungarian Empire and
after World War I became the capital of Czechoslovakia. The city played major
roles in the Protestant Reformation, the Thirty Years' War, and in 20th-century
history, during both World Wars and the post-war Communist era.[citation
needed]

Prague is home to a number of famous cultural attractions, many of which
survived the violence and destruction of 20th-century Europe. Main attractions
include the Prague Castle, the Charles Bridge, Old Town Square with the Prague
astronomical clock, the Jewish Quarter, Petřín hill and Vyšehrad. Since 1992,
the extensive historic centre of Prague has been included in the UNESCO list of
World Heritage Sites.

The city boasts more than ten major museums, along with numerous theatres,
galleries, cinemas, and other historical exhibits. An extensive modern public
transportation system connects the city. Also, it is home to a wide range of
public and private schools, including Charles University (Univerzita Karlova v
Praze). Prague is classified as an "Alpha-" global city according to GaWC
studies, comparable to Vienna, Seoul and Washington, D.C. Its rich history
makes it a popular tourist destination, and the city receives more than 4.4
million international visitors annually, as of 2011. Prague ranked fifth in
the Tripadvisor world list of best destinations in 2014. Prague is the
fifth most visited European city after London, Paris, Istanbul and Rome.

\end{document}
